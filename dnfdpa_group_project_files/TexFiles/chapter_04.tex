\chapter{Literature résumés}
\setlength{\headheight}{12.71342pt}
\addtolength{\topmargin}{-0.71342pt}

This section of the course notes is designed to streamline access to the key findings from each reading material (RM), providing a concise and accessible overview of essential information. Created through experimentation with various AI platforms, this chapter also serves to enhance prompt engineering skills, exploring diverse methods of note-taking for maximum efficiency and clarity. The procedures for creating these summaries have varied, but all methods share a common approach: each RM has been fully read, with summaries and notes prepared after completing each respective subsection. By using these AI-co-op'ed approaches, these notes aim to be both a reliable reference and a resource for continuous improvement in capturing complex microbiology concepts.

\section{1\texorpdfstring{\textsuperscript{st}}{st} Reading Material from the Curriculum}

\subsection{Milk for Liquid Consumption}
Liquid milk is treated by pasteurization or sterilization to ensure safety, extend shelf life, and retain flavour. Raw milk is considered unsafe and is restricted in many countries. Pasteurized milk retains better flavour, while sterilized milk offers longer shelf life, especially valued in cooking. Fat content is usually standardized, but low-fat and skim milks are also common. Some products are fortified or processed via ultrafiltration for consistent protein content, although this may be legally restricted. Quality attributes vary by use, and packaging is essential for hygiene \cite*{curr_rm_01_dairy_science_technology}.

\subsubsection*{Manufacture}
Thermalization reduces lipase activity and psychrotrophic growth, aiding shelf life. Homogenization prevents creaming but increases lipolysis risk, requiring higher heat (e.g., 20 s at 75~\textdegree C). Low pasteurization (15 s at 72~\textdegree C) kills pathogens while preserving natural inhibitors, though heat-sensitive compounds like agglutinins and immunoglobulins are degraded in high-pasteurized milk. Packaging hygiene is crucial to prevent recontamination and preserve shelf life \cite*{curr_rm_01_dairy_science_technology}.


\subsubsection*{Shel Life}
Shelf life is influenced by bacterial growth, enzymatic activity, and chemical/physical changes. Key factors include storage temperature, recontamination, and Bacillus cereus spore levels. Below 7~\textdegree C, psychrotrophs dominate spoilage. Hygiene in packaging is essential; rapid tests help detect recontamination \cite*{curr_rm_01_dairy_science_technology}.

\subsubsection*{Extended-Shelf-Life Milk}
ESL milk combines long shelf life with near-fresh flavour. One method applies short-time direct UHT treatment (e.g., 2 s at 140~\textdegree C) with aseptic packaging; enzymes like plasmin may still affect taste after weeks. The second method involves microbial removal via microfiltration or bactofugation, often followed by partial UHT sterilization of retentate and cream. Aseptic packaging is essential. Cooked flavour is minimized by limiting heat to fat-rich fractions \cite*{curr_rm_01_dairy_science_technology}. 

\subsection{Sterilized Milk}
\subsubsection{Description}
Sterilized milk must be microbe-free, shelf-stable at ambient temperature, and retain acceptable flavour and nutritional value. UHT sterilization (e.g., 1 s at 145~\textdegree C) minimizes browning, off-flavours, and vitamin loss. To prevent spoilage, packaging must be aseptic, and milk free from heat-resistant enzymes. Homogenization avoids creaming and coalescence. Lactulose content is used to identify UHT-treated milk \cite*{curr_rm_01_dairy_science_technology}.

\subsubsection*{Manufacture}
Sterilized milk is made via in-bottle, mild in-bottle, or flow-through UHT processes. Psychrotroph enzymes (esp. from \textit{Pseudomonas}) are heat-resistant, so raw milk must be fresh. UHT heating (>140~\textdegree C) ensures safety but risks casein aggregation, off-flavors, and vitamin loss. Aseptic homogenization and deaeration are crucial to prevent oxidized flavor. Oxygen- and light-tight packaging prolongs shelf life \cite*{curr_rm_01_dairy_science_technology}. 

\subsubsection*{Shelf Life}
Spoilage of in-bottle sterilized milk may result from surviving spores (e.g., \textit{B. subtilis}, \textit{B. stearothermophilus}) or leaky packaging. UHT milk mainly deteriorates via recontamination or residual heat-resistant enzymes, causing gelation, off-flavors, or plasmin-induced bitterness. Nonenzymatic spoilage includes oxidation, Maillard reactions, and light effects. Shelf life is tested via incubation, oxygen pressure, or ATP bioluminescence \cite*{curr_rm_01_dairy_science_technology}.

\subsection{Reconstituted Milk}
Reconstituted milk is made by dissolving milk powder in water; recombined milk adds anhydrous milk fat to reconstituted skim milk. It mimics whole milk but lacks natural fat globule membrane components. Filled milk uses vegetable oil instead of milk fat. Toned milk blends buffalo milk with skim milk to reduce fat content \cite*{curr_rm_01_dairy_science_technology}. 

\subsection{Flavour}
Good flavour means a bland taste without off-flavours. Sources include microbial growth (e.g., \textit{B. cereus}, \textit{P. fragii}), plasmin, lipoprotein lipase, and oxidation by Cu or light. Heat causes cooked, UHT ketone, or sterilized-milk flavour, depending on thermal load. Sunlight flavour arises from methionine oxidation with riboflavin present \cite*{curr_rm_01_dairy_science_technology}.

\subsection{Nutritive Value}
This section addresses changes in nutritive value due to deliberate changes in composition, processing, and storage. For details on the nutritive aspects of milk components, see Subsections 2.1.2, 2.2.4, 2.3.3, 2.4.5, and Table 2.18 in the book \cite*{curr_rm_01_dairy_science_technology}.

\subsubsection*{Modification of Composition}
Milk of modified composition includes low-fat, skim, or vitamin-fortified types. Filled milk uses vegetable oils, often rich in vitamins D and E, with added antioxidants. Calcium may be added as lactate or whey permeate. Lactose-free milk, produced by adding lactase after UHT treatment, has limited success due to cost and sweet taste. Functional foods and specialised products are also being developed \cite*{curr_rm_01_dairy_science_technology}.

\subsubsection*{Loss of Nutrients}
Pasteurized and UHT-sterilized milk lose few nutrients, while in-bottle sterilized milk shows greater loss, especially of lysine and vitamins due to Maillard reactions. Losses mainly affect vitamin C and B vitamins ($B_1$, $B_2$, $B_6$, $B_9$, $B_12$). Oxygen and light accelerate degradation, with riboflavin acting as a catalyst. Packaging permeability is crucial to prevent losses \cite*{curr_rm_01_dairy_science_technology}.

\subsection{Infant Formulas}
Breast feeding is preferable, but when not possible, infant formulas based on cows' milk fractions are used. Unmodified cows' milk is unsuitable. Due to higher risk of microbial contamination, strict hygiene is essential during preparation and storage. Liquid formulas should be refrigerated \cite*{curr_rm_01_dairy_science_technology}.

\subsubsection*{Human Milk}
Human milk differs from cows’ milk in composition and varies by individual and lactation stage. It contains more essential fatty acids, cholesterol, and oligosaccharides, but less protein, casein, and minerals. It includes immunoglobulin A, lysozyme, and lactoferrin, and lacks $\beta$-lactoglobulin. Infant formulas require significant adjustment to mimic its properties \cite*{curr_rm_01_dairy_science_technology}.

\subsubsection*{Formula Composition and Manufacture}
Infant formulas use skim milk and sweet whey (e.g., 1:5 ratio), often with added lactose, vegetable oils, vitamins, Fe, and Cu. Whey is partly desalted. Oligosaccharides or lactulose may be added. Manufacture involves wet mixing, pre-emulsification, pasteurization, and homogenization. Products may be UHT-sterilized, canned, or spray-dried \cite*{curr_rm_01_dairy_science_technology}.


