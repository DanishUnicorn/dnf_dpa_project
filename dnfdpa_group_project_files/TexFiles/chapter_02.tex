\setcounter{chapter}{0}
\setcounter{section}{0}
\chapter{Lecture Notes}
\setlength{\headheight}{12.71342pt}
\addtolength{\topmargin}{-0.71342pt}

\section{Lecture 01 - 04/08-2025}
\subsection*{Topics of the course lectures}
The course will be covering the following topics:
\begin{itemize}
    \item Week 1
    \subitem Reading materials for self-study and start the project work
    \item Week 2
    \subitem Monday 4 August: Introduction to the course and Basic constituents of milk and alternatives (Lilia, 
    \subitem Jorg) + project work
    \subitem Tuesday 5 August: Dairy processing (Jorg) and Introduction and Sustainable Nutrition (Merete \& Lea)
    \subitem Wednesday 6 August: Digestion and Nutritional effects of dairy protein (Lilia, Jorg, Mads) + project \subitem work
    \subitem Thursday 7 August: Lactose, calcium and lipids (Lilia, Jorg, Andreas) + project work / Pitch your idea
    \subitem Friday 8 August: The challenge: sustainable and healthy diets - from purpose to practice (Thom, Nick \subitem Smith) + project work
    \item Week 3
    \subitem Monday 11 August: Process-induced changes (Qing Ren)
    \subitem Tuesday 12 August: Process-induced changes (Qing Ren, Lilia)
    \subitem Wednesday 13 August: Dairy/Plant-Hybrid Products (Iben, Lilia)
    \subitem Thursday 14 August: Human Nutrition (Inge, Marta, Jordi)
    \subitem Friday 15 August: Gut Microbiota (Torben)
    \item Week 4 \& 5
    \subitem Project work
\end{itemize}

\subsection*{Structure of the group project}
The structure of the report was defined and must adhere to the following:
\begin{itemize}
    \item 30 +/- 10 pages
    \item Product description, including the purpose of “why” the product
    \item Intended nutrition or health benefits
    \item Description of the ingredients
    \item Description of the claimed beneficial effect
    \item Target consumer group
    \item Dietary pattern of the chosen consumer group, and how the product would fit in that
    \item Formulation - which raw materials will be used, and why. In case of ingredients, which processes would they already have undergone prior to product formulation?
    \item Processing - how will the product be processed? Process flow diagram and identify the impact of processing steps on nutritional value (could it impact the nutrition/health benefits)
    \item Nutritional Quality - To quantitatively or qualitatively estimate the nutritional and sustainability aspects of the novel product
\end{itemize}

\section{Lecture 02 - 04/08-2025}

The physical properties of milk are defined by:
\begin{table}[h]
    \centering
    \caption{Examples of physical properties of milk}
    \label{tab:milk_physical_properties}
    \rowcolors{2}{white}{gray!7}
    \begin{tabular}{ l  r l }
        \textbf{Property} & \textbf{Value} & \textbf{Unit}\\ 
        \hline
        Osmotic Pressure & $= 700$  & $[kPa]$ \\ 

        Freezing-Point Depression & $\approx 0.54$ & $[K]$ \\

        Electrical Conductivity & $\approx 0.5$ & $[A/(V \cdot m)]$ \\ 

        pH Value & $6.7$ & \\

        Ioinic Strength & $\approx 0.08$ & $[molar]$ \\

        Water Activity & $\approx 0.993$ & \\

    \end{tabular}
\end{table}

\section{Lecture 03 - 05/08-2025}




