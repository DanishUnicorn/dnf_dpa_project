\chapter*{Preface and abstract}
\setlength{\headheight}{12.71342pt}
\addtolength{\topmargin}{-0.71342pt}


\section*{Preface}
This written assignment has been prepared as part of the course NFOK24004U - Dairy and Plant-based Alternatives at the University of Copenhagen. The course addresses the challenges of developing sustainable and healthy diets by examining processing effects on nutrients in dairy products, hybrid products and their alternatives. 

\vspace{1em}
The project is a theoretical study on the development of a hemp seed protein bar as a dairy substitute. Through this work, we aimed to apply the knowledge and competences obtained during the course, including nutritional evaluation, processing considerations, sustainability aspects, and consumer perspectives. The assignment was carried out by Sofie Karoline Thue Hansen (FVC568), Nils Hugo Nilsson (XQK212), Niclas Hauerberg Hyldahl (JNC117), and Lucas Daniel Paz Zuleta (TZS159), all MSc students at the University of Copenhagen.

\section*{Abstract}
This project explores the development of a hemp seed protein bar as a sustainable alternative to dairy-based protein products. The aim was to design a nutrient-rich product with a favourable environmental profile, while addressing consumer demand for plant-based, convenient, and health-oriented snacks. The nutritional composition was assessed through literature-based data on macronutrients, dietary fibres, and fatty acids, with focus on protein quality and digestibility. Comparisons were made to existing market products (ROO'bar hemp protein bar), highlighting the bar's potential for high protein and high fibre claims under EU regulations. The lipid fraction showed a desirable omega-6 to omega-3 ratio, although thresholds for authorised health claims were not met. Environmental perspectives further emphasised the advantages of hemp cultivation, including low carbon footprint, soil health benefits, and potential use of side streams. Together, these findings demonstrate the relevance of hemp seeds in developing innovative plant-based products that align with both nutritional and sustainability goals.

\newpage
