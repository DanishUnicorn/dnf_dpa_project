
\section{Rules of probability}

\subsection{Probability density function}

The probability density function (pdf) of a random variable $X$ is a function $f(x)$ that describes the likelihood of observing a value $x$ of $X$.

\begin{align}
    f(x) = P(X=x), \quad 
    \text{where } x \in \text{set of possible values of } X.
\end{align}

It does not make sense to talk about the pdf at a specific value for a continuous random variable (as this will always be zero). Instead, we talk about the probability of observing a value in an interval.

\subsection{Cumulative density function}

The cumulative density function (cdf) of a random variable $X$ is a function $F(x)$ that describes the probability of observing a value less than or equal to $x$.

\begin{equation}
    F(x) = P(X \leq x)  
    \quad \text{where } x \in \text{set of possible values of } X.
\end{equation}

\subsection{Link between CDF and PDF}

The pdf and cdf are linked by the following equation

\begin{equation}
    p(a \leq X \leq b) = F(b) - F(a),
\end{equation}

where $b > a$. 

\subsection{Complement rule}

The probability of an event not happening is one minus the probability of the event happening. 

\begin{equation}
    p(\bar{A}) = 1 - p(A)
\end{equation}

This can be used in the following way:

\begin{equation}
    p(X > x) = 1 - p(X \leq x) = 1 - F(x)
\end{equation}
    
\section{Rules of random variables}

\subsection{Mean rule}

Let $X$ be a random variable and $a$ and $b$ be constants $\in \R$ then

\begin{equation}
    \E(aX + b) = a\E(X) + b.
\end{equation}

\subsection{Variance rule}

Let $X$ be a random variable and $a$ and $b$ be constants $\in \R$ then

\begin{equation}
    \text{Var}(aX + b) = a^2 \text{Var}(X).
\end{equation}

\subsection{Mean of linear combinations}

Let $X_1, ... , X_n$ be \textcolor{red}{independent} random variables then

\begin{equation}
    \E(aX_1 + bX_2 + ... + cX_n) = a\E(X_1) + b\E(X_2) + ... + c\E(X_n).
\end{equation}

\subsection{Variance of linear combinations}

Let $X_1, ... , X_n$ be \textcolor{red}{independent} random variables then

\begin{equation}
    \text{Var}(aX_1 + bX_2 + ... + cX_n) = a^2\text{Var}(X_1) + b^2\text{Var}(X_2) + ... + c^2\text{Var}(X_n).
\end{equation}

\section{Discrete distributions}


\subsection{Binomial distribution}

Let the random variable $X$ be the number of successes in independent $n$ draws with replacement. Then $X$ follows the binomial distribution

\begin{equation}
X \sim B(n,p),
\end{equation}

where $n$ is the number of draws and $p$ is the probability of success in each trial. The binomial pdf describes the probability of obtaining $x$ successes in $n$ draws

\begin{equation}
f(x;n,p) = P(X=x) = \binom{n}{x} p^x (1-p)^{n-x},
\end{equation}

where,

\begin{equation}
\binom{n}{x} = \frac{n!}{x!(n-x)!},
\end{equation}

is the number of distinct sets of $x$ elements which can be chosen from a set of $n$ elements. Remember that $n! = n \cdot (n-1) \cdot (n-2) \cdot ... \cdot 2 \cdot 1$.

The mean of a binomial distributed random variable is

\begin{equation}
\mu = np,
\end{equation}
and the variance is,
\begin{equation}
\sigma^2 = np(1-p).
\end{equation}


\begin{highlight}
    The binomial distribution has the following assumptions:
\begin{itemize}
    \item The draws are independent.
    \item Each draw is made with replacement. That is, the probability of success does not change from draw to draw.
    \item If the set from where the "draws" are taken is approximately infinite, the binomial distribution can be used without replacement. 
\end{itemize}
\end{highlight}


\subsection{Hypergeometric distribution}

Let the random variable $X$ be the number of successes in $n$ draws without replacement. Then $X$ follows the hypergeometric distribution

\begin{equation}
X \sim H(n, a, N),
\end{equation}

where $a$ is the number of successes in the population and $N$ is the population size. The hypergeometric pdf describes the probability of obtaining $x$ successes in $n$ draws

\begin{equation}
f(x;n,a,N) = P(X=x) = \frac{\binom{a}{x} \binom{N-a}{n-x}}{\binom{N}{n}},
\end{equation}

where,

\begin{equation}
\binom{a}{b} = \frac{a!}{b!(a-b)!},
\end{equation}

is the number of distinct sets of $b$ elements which can be chosen from a set of $a$ elements. 

The mean of a hypergeometric distributed random variable is

\begin{equation}
\mu = \frac{na}{N},
\end{equation}

and the variance is,

\begin{equation}
\sigma^2 = n\frac{a(N-a)}{N^2} \cdot \frac{N-n}{N-1}
\end{equation}

\subsection{Poisson distribution}

Let the random variable $X$ be the number of events in a given interval (time, space or other). Then $X$ follows the Poisson distribution

\begin{equation}
X \sim Po(\lambda),
\end{equation}

where $\lambda$ is the average number of events in the interval (called rate or intensity). The Poisson pdf describes the probability of observing $x$ events in an interval

\begin{equation}
f(x;\lambda) = P(X=x) = \frac{\lambda^x}{x!} e^{-\lambda},
\end{equation}

The mean of a Poisson distributed random variable is
\begin{equation}
\mu = \lambda,
\end{equation}
and the variance is,
\begin{equation}
\sigma^2 = \lambda.
\end{equation}

\subsubsection{Rate scaling of Poisson distribution}

The Poisson distribution is often used to model the number of events in a given interval. If the interval is scaled by a factor $c$, the rate of events will also scale by $c$. This means that if $X \sim Po(\lambda)$, then $cX \sim Po(c\lambda)$.

\begin{example}{Rate scaling}{rate_scaling}
If a 10 mL sample has 5 bacteria on average,
\begin{equation*}
X \sim Po(\lambda = 5),
\end{equation*}

Then we can find the rate for a 20 mL sample by scaling the rate by the factor $c = 2$,

\begin{equation*}
cX \sim Po(c\lambda) = Po(2 \cdot 5) = Po(10).
\end{equation*}
\end{example}

\section{Color codes for highlighting}
Apricot
Aquamarine
Bittersweet
BlueGreen
BrickRed
BurntOrange
CarnationPink
Cerulean
CornflowerBlue
Dandelion
Emerald
ForestGreen
Fuchsia
Goldenrod
Lavender
LimeGreen
Magenta
Melon
MidnightBlue
Mulberry
NavyBlue
OliveGreen
Orange
Orchid
Peach
Periwinkle
PineGreen
Plum
RawSienna
RedOrange
RoyalBlue
RubineRed
SeaGreen
SkyBlue
SpringGreen
Tan
TealBlue
Thistle
Turquoise
Violet
WildStrawberry
YellowGreen





